\section{Exercise with Granta}

\textcolor{red}{TRADUCI TUTTO IN INGLESE}

\paragraph{Overview:} L'oggetto a noi assegnato è l'elmetto da bici (più precisamente il guscio esterno). Ecco riportato in ordine il processo di selezione dei due miglior materiali candidati:

\begin{itemize}
    \item Selezione dei material indices dalla seguente \href{https://www.ansys.com/content/dam/amp/2021/august/webpage-requests/education-resources-dam-upload-batch-6/performance-indices-booklet-bokpeien22.pdf}{\textit{tabella}}.
    \item Applicazione dei limiti
    \item Applicazione di un material index (costo)
    \item Selezione del materiale più cheap
    \item Rimozione del material index relativo al costo e selezione del material più expensive
\end{itemize}

\paragraph{Material indeces:} Prima di tutto scegliamo la geometria che meglio approssima l'oggetto finale che dobbiamo costruire. Successivamente ci siamo basati su Stiffness-limited-Design at minimum mass.

\begin{figure}[H]
    \centering
    \includegraphics[width=\linewidth]{images/dome_data.png}
    \caption{Function and Constraints and formula to maximize}
    \label{fig:dome_1}
\end{figure}

\noindent Successivamente massimizziamo e Strength-limited-Design at minimum mass:

\begin{figure}[H]
    \centering
    \includegraphics[width=\linewidth]{images/dome_data2.png}
    \caption{Function and Constraints and formula to maximize}
    \label{fig:dome_2}
\end{figure}

\noindent I seguenti Ashby chart:

\begin{figure}[H]
    \centering
    \includegraphics[width=\linewidth]{images/mappa stifness comp.JPG}
    \caption{Stiffness-limited-Design at minimum mass}
    \label{fig:ashby_chart_1}
\end{figure}

\begin{figure}[H]
    \centering
    \includegraphics[width=\linewidth]{images/mappa strenght comp.JPG}
    \caption{Strength-limited-Design at minimum mass}
    \label{fig:ashby_chart_2}
\end{figure}

Applicando la formula dei material indices, per ogni grafico abbiamo posizionato una retta di pendenza 1 in modo da restringere il numero di candidates materials e massimizzare le proprietà.

\begin{figure}[H]
    \centering
    \includegraphics[width=\linewidth]{images/Stiffness.JPG}
    \caption{Maximizing Stiffness-limited-Design at minimum mass}
    \label{fig:min_ashby_chart_1}
\end{figure}

\begin{figure}[H]
    \centering
    \includegraphics[width=\linewidth]{images/Strength.JPG}
    \caption{Maximizing Strength-limited-Design at minimum mass}
    \label{fig:min_ashby_chart_2}
\end{figure}

\noindent Come si può vedere dai grafici le rette sono state posizionate in modo da mediare sulla selezione delle proprietà.

\paragraph{Costi:} Abbiamo impostato un limite sulla geometria del nostro materiale in modo da avere una geometria il più simile ad un casco da bici, come riportato in Fig. \ref{fig:forma}. Questo limite permette al programma di selezionare tutti i materiale che possono ottenere la forma da noi desiderata tramite vari processi industriali.

\begin{figure}[H]
    \centering
    \includegraphics[width=\linewidth]{images/forma.JPG}
    \caption{Selection of the closest resembling geometrical shape}
    \label{fig:forma}
\end{figure}

\paragraph{Limits:} Abbiamo impostato il material index basato sul costo (vedi Fig. \ref{fig:dome_3}):

\begin{figure}[H]
    \centering
    \includegraphics[width=\linewidth]{images/dome_data3.png}
    \caption{fra non ha voglia....}
    \label{fig:dome_3}
\end{figure}

\noindent Il seguenti Ashby chart:

\begin{figure}[H]
    \centering
    \includegraphics[width=\linewidth]{images/costo.JPG}
    \caption{fra non ha voglia....}
    \label{fig:costo}
\end{figure}

Questa volta abbiamo posizionato la retta in modo da selezionare un solo materiale candidato, che corrisponde all'opzione più economica del casco, come riportato in Fig. \ref{fig:cip_ciop}

\begin{figure}[H]
    \centering
    \includegraphics[width=\linewidth]{images/cheap chop.JPG}
    \caption{Cheap chop}
    \label{fig:cip_ciop}
\end{figure}

\noindent Il materiale trovato è \texttt{Fir (abies lasiocarpa) (longitudinal)}

\begin{figure}[H]
    \centering
    \includegraphics[width=\linewidth]{images/material_data_fir.JPG}
    \caption{Fir footprint}
    \label{fig:fir_footprint}
\end{figure}

\paragraph{Selezione del materiale più expensive:} Abbiamo rimosso il material index relativo al costo in modo da poter selezionare un materiale che massimizzi lo Stiffness index. Inoltre abbiamo applicato dei limiti per migliorare la durability: resistenza all'acqua, alla luce solare e infiammabilità.

\begin{figure}[H]
    \centering
    \includegraphics[width=\linewidth]{images/filter.JPG}
    \caption{Selection of the closest resembling geometrical shape}
    \label{fig:filtro}
\end{figure}

\begin{figure}[H]
    \centering
    \includegraphics[width=\linewidth]{images/missing.PNG}
    \caption{Expensive material}
    \label{fig:ashby_map_missing}
\end{figure}


\paragraph{Ecoaudit} 


\begin{figure}[H]
    \centering
    \includegraphics[width=\linewidth]{images/ecoaudit_project_carbon_fiber.JPG}
    \caption{Eco audit carbon fiber}
    \label{fig:ecoaudit_carbon}
\end{figure}

\begin{figure}[H]
    \centering
    \includegraphics[width=\linewidth]{images/ecoaudit_project_fir.JPG}
    \caption{Eco audit fir}
    \label{fig:ecoaudit_fir}
\end{figure}

\paragraph{Environmental impact analysis} Abbiamo performato un eco audit per valutare impatto ecologico generato dai materiali selezionati, comparando il materiale meno costoso (Fir) con quello più costoso (Cyanate ester/HM carbon fiber). 

Per far ciò abbiamo dovuto compilare le varie categorie inserendo i valori del “Fir (abies lasiocarpa) (l) e del Cyanate ester/HM carbon fiber nell'Eco Audit Projet, tenendo in considerazione i parametri di riciclaggio e fine vita presenti nel database dei due materiali (figure sotto).

\begin{figure}[H]
    \centering
    \includegraphics[width=\linewidth]{images/material_data_fir_crop.JPG}
    \caption{Parametri fir}
    \label{fig:params_fir}
\end{figure}

\begin{figure}[H]
    \centering
    \includegraphics[width=\linewidth]{images/material_data_Cy_crop.JPG}
    \caption{Parametri Cy}
    \label{fig:params_cy}
\end{figure}

Nella sezione “Material, manufacture and end of life” abbiamo considerando una quantità di 1000 elmetti, senza contenuto di riciclaggio ed una massa di 0,15 kg (ChatGPT). Per la End of life del materiale è stato scelto Downcycle per entrambi in quanto i due materiali non possono essere riciclati.

È stata inoltre selezionata la modalità di trasporto e la distanza da percorrere per la distribuzione del prodotto (Vedi fig. \ref{fig:ecoaudit_fir} \& \ref{fig:ecoaudit_carbon})

Tenendo in considerazione la durability dei materiali, abbiamo scelto di considerare una product life diversa per i due prodotti (5 anni per quello più economico e 10 per quello più costoso). Vedi fig. \ref{fig:ecoaudit_fir} nella sezione \texttt{Use}.

Otteniamo i seguenti summary chart:

\begin{figure}[H]
    \centering
    \includegraphics[width=\linewidth]{images/energy_char.png}
    \caption{Energy chart}
    \label{fig:energy_char}
\end{figure}

\begin{figure}[H]
    \centering
    \includegraphics[width=\linewidth]{images/co2_char.png}
    \caption{CO2 chart}
    \label{fig:co2_char}
\end{figure}

Dai due grafici possiamo dedurre che per nessuno dei due prodotti è possibile compensare la CO2 generata nei processi produttivi. Questo Il Cyanate dal punto di vista della sostenibilità è peggiore rispetto al Fir, sia in termini di energia necessaria alla sua produzione che emissioni di CO2. Il Fir pur non avendo durability e proprietà meccaniche migliori, ha vantaggi maggiori in termini di sostenibilità in quanto è biodegradabile e non produce quantità rilevanti di CO2 nella produzione. 