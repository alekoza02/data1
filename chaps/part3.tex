\section{Exercise with Granta}

\subsection{Material selection}

\paragraph{Overview:} The object assigned to us is the bicycle helmet (more precisely, the outer shell). Here is the selection process for the two best candidate materials, in order:

\begin{itemize}
    \item Selection of material indices from the following \href{https://www.ansys.com/content/dam/amp/2021/august/webpage-requests/education-resources-dam-upload-batch-6/performance-indices-booklet-bokpeien22.pdf}{\textit{table}}.
    \item Application of limits
    \item Application of a material index (cost)
    \item Selection of the cheapest material
    \item Removal of the material index relating to cost and selection of the most expensive material
\end{itemize}

\paragraph{Material indeces:} First of all, we choose the geometry that best approximates the final object we need to build. Next, we based our design on Stiffness-limited-Design at minimum mass.

\begin{figure}[H]
    \centering
    \includegraphics[width=\linewidth]{images/dome_data.png}
    \caption{Function and Constraints and formula to maximize}
    \label{fig:dome_1}
\end{figure}

\noindent Next, we maximize and Strength-limited-Design at minimum mass:

\begin{figure}[H]
    \centering
    \includegraphics[width=\linewidth]{images/dome_data2.png}
    \caption{Function and Constraints and formula to maximize}
    \label{fig:dome_2}
\end{figure}

\noindent Ashby charts:

\begin{figure}[H]
    \centering
    \includegraphics[width=\linewidth]{images/mappa stifness comp.JPG}
    \caption{Stiffness-limited-Design at minimum mass}
    \label{fig:ashby_chart_1}
\end{figure}

\begin{figure}[H]
    \centering
    \includegraphics[width=\linewidth]{images/mappa strenght comp.JPG}
    \caption{Strength-limited-Design at minimum mass}
    \label{fig:ashby_chart_2}
\end{figure}

Applying the material indices formula, we positioned a straight line with a slope of 1 on each graph in order to narrow down the number of candidate materials and maximize properties.

\begin{figure}[H]
    \centering
    \includegraphics[width=\linewidth]{images/Stiffness.JPG}
    \caption{Maximizing Stiffness-limited-Design at minimum mass}
    \label{fig:min_ashby_chart_1}
\end{figure}

\begin{figure}[H]
    \centering
    \includegraphics[width=\linewidth]{images/Strength.JPG}
    \caption{Maximizing Strength-limited-Design at minimum mass}
    \label{fig:min_ashby_chart_2}
\end{figure}

\noindent As can be seen from the graphs, the lines have been positioned so as to mediate on the selection of properties.

\paragraph{Limit:} We have set a limit on the geometry of our material so that it is as similar as possible to a bicycle helmet, as shown in Fig. \ref{fig:forma}. This limit allows the program to select all materials that can achieve the desired shape through various industrial processes.

\begin{figure}[H]
    \centering
    \includegraphics[width=\linewidth]{images/forma.JPG}
    \caption{Selection of the closest resembling geometrical shape}
    \label{fig:forma}
\end{figure}

\paragraph{Cost:} We set up the material index based on cost (see Fig. \ref{fig:dome_3}):

\begin{figure}[H]
    \centering
    \includegraphics[width=\linewidth]{images/dome_data3.png}
    \caption{Mechanical - Strength table}
    \label{fig:dome_3}
\end{figure}

\begin{figure}[H]
    \centering
    \includegraphics[width=\linewidth]{images/costo.JPG}
    \caption{Price vs Flexural modulus Ashby chart}
    \label{fig:costo}
\end{figure}

This time, we positioned the line so as to select only one candidate material, which corresponds to the most economical helmet option, as shown in Fig. \ref{fig:cip_ciop}.

\begin{figure}[H]
    \centering
    \includegraphics[width=\linewidth]{images/cheap chop.JPG}
    \caption{Candidate as economic material}
    \label{fig:cip_ciop}
\end{figure}

\noindent The material we found is \texttt{Fir (abies lasiocarpa) (longitudinal)}

\begin{figure}[H]
    \centering
    \includegraphics[width=\linewidth]{images/material_data_fir.JPG}
    \caption{Fir footprint}
    \label{fig:fir_footprint}
\end{figure}

\paragraph{Selection of the most expensive material:} We removed the material index related to cost so that we could select a material that maximizes the stiffness index. We also applied limits to improve durability: resistance to water, sunlight, and flammability.

\begin{figure}[H]
    \centering
    \includegraphics[width=\linewidth]{images/filter.JPG}
    \caption{Selection of the closest resembling geometrical shape}
    \label{fig:filtro}
\end{figure}

\begin{figure}[H]
    \centering
    \includegraphics[width=\linewidth]{images/missing.PNG}
    \caption{Expensive material}
    \label{fig:ashby_map_missing}
\end{figure}


\subsection{Ecoaudit} 

\paragraph{Environmental impact analysis} We performed an eco-audit to assess the ecological impact generated by the selected materials, comparing the least expensive material (Fir) with the most expensive (Cyanate ester/HM carbon fiber).

\begin{figure}[H]
    \centering
    \includegraphics[width=\linewidth]{images/ecoaudit_project_carbon_fiber.JPG}
    \caption{Eco audit carbon fiber}
    \label{fig:ecoaudit_carbon}
\end{figure}

\begin{figure}[H]
    \centering
    \includegraphics[width=\linewidth]{images/ecoaudit_project_fir.JPG}
    \caption{Eco audit fir}
    \label{fig:ecoaudit_fir}
\end{figure}

To do this, we had to compile the various categories by entering the values for Fir (Abies lasiocarpa) (l) and Cyanate ester/HM carbon fiber in the Eco Audit Project, taking into account the recycling and end-of-life parameters present in the database for the two materials (figures below).

\begin{figure}[H]
    \centering
    \includegraphics[width=\linewidth]{images/material_data_fir_crop.JPG}
    \caption{Fir parameters}
    \label{fig:params_fir}
\end{figure}

\begin{figure}[H]
    \centering
    \includegraphics[width=\linewidth]{images/material_data_Cy_crop.JPG}
    \caption{Cy parameters}
    \label{fig:params_cy}
\end{figure}

In the program section “Material, manufacture, and end of life” we considered a quantity of 1000 helmets, with no recycled content and an estimated mass of 0.15 kg. For the end of life of the material, Downcycle was chosen for both, as the two materials cannot be recycled.

The mode of transport and distance to be traveled for product distribution were also selected (see fig. \ref{fig:ecoaudit_fir} \& \ref{fig:ecoaudit_carbon}).

Taking into account the durability of the materials, we chose to consider a different product life for the two products (5 years for the cheaper one and 10 for the more expensive one). See fig. \ref{fig:ecoaudit_fir} in the \texttt{Use} section.

We thus obtain the following summary chart:

\begin{figure}[H]
    \centering
    \includegraphics[width=\linewidth]{images/energy_char.png}
    \caption{Energy chart}
    \label{fig:energy_char}
\end{figure}

\begin{figure}[H]
    \centering
    \includegraphics[width=\linewidth]{images/co2_char.png}
    \caption{$CO_2$ chart}
    \label{fig:co2_char}
\end{figure}

From the two graphs, we can deduce that for neither product is it possible to offset the $CO_2$ generated in the production processes. From a sustainability perspective, cyanate is worse than FIR, both in terms of the energy required for its production and $CO_2$ emissions. Although FIR does not have better durability and mechanical properties, it has greater advantages in terms of sustainability as it is biodegradable and does not produce significant amounts of $CO_2$ during production.