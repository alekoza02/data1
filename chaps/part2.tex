\section{Exercises with PANDAT}

\subsection{Exercise 2}
\begin{figure}[H]
    \centering
    \includegraphics[width=\linewidth]{images/Cu-Ni Phase diagram.jpg}
    \caption{Phase diagram Cu-Ni}
    \label{fig:dario6}
\end{figure}

In this diagram we have a Cu-Ni binary phase diagram, which is an Isomorphous system, so copper and nickel are completely soluble in each other in both liquid and solid states.
We can see 3 distinct regions representing different phase states: 

\begin{itemize}
    \item Liquid: a single liquid solution phase where Cu and Ni are fully mixed. 
    \item $\alpha$ (Solid Solution); 
    \item Liquid + $\alpha$: the region between the two curves where both liquid and solid coexist.
\end{itemize}

\paragraph{Invariant Points:}
There are no invariant points, such as eutectic or eutectoid points in the diagram. In this isomorphous system, the only fixed points are the melting temperatures of the two pure metals.

With the Lever Rule, we can calculate the fraction of the two phases. The fraction of one phase is proportional to the length of the tie line segment on the opposite side of the overall composition point.

\begin{figure}[H]
    \centering
    \includegraphics[width=\linewidth]{images/Leva_edited.jpg}
    \caption{Lever rule on Cu-Ni}
    \label{fig:leva}
\end{figure}

\noindent \textbf{Fraction of each phase: }
\begin{itemize}
    \item Liquid: $F_L = \frac{S}{R + S} = \frac{C_{\alpha} - C_0}{C_{\alpha} - C_L} = 0.333$
    \item $\alpha$: $F_{\alpha} = \frac{R}{R + S} = \frac{C_0 - C_L}{C_{\alpha} - C_L} = 0.667$
\end{itemize}

\noindent \textbf{Using the point calculation we obtain:}
\begin{itemize}
    \item Liquid: 0.38
    \item $\alpha$: 0.62
\end{itemize}


\newpage

% \begin{figure}[H]
%     \centering
%     \includegraphics[width=\linewidth]{images/cuni1500.jpg}
%     \caption{Energy curve @ 1500°C}
%     \label{fig:dario7}
% \end{figure}
\begin{figure}[H]
    \centering
    \includegraphics[width=\linewidth]{images/cuni1400_comp.jpg}
    \caption{Tangent method @ 1400°C}
    \label{fig:dario1_comp}
\end{figure}
\begin{figure}[H]
    \centering
    \includegraphics[width=\linewidth]{images/cuni1300_comp.jpg}
    \caption{Tangent method @ 1300°C}
    \label{fig:dario2_comp}
\end{figure}
\begin{figure}[H]
    \centering
    \includegraphics[width=\linewidth]{images/cuni1200_comp.jpg}
    \caption{Tangent method @ 1200°C}
    \label{fig:dario5_comp}
\end{figure}
\begin{figure}[H]
    \centering
    \includegraphics[width=\linewidth]{images/cuni1100_comp.jpg}
    \caption{Tangent method @ 1100°C}
    \label{fig:dario3_comp}
\end{figure}

\newpage

\noindent In the free energy curve diagram, we can see 2 curves, G(@Liquid) and G(@Fcc). When these curves intersect, there is a region where the system can minimize its energy by following a straight line that is tangent to both curves. This line is representing the lowest possible free energy for the system and the 2 points where this line touches the curves define the equilibrium composition.
$X^L$ is the composition of the liquid phase and $X^{\alpha}$ is the composition of the $\alpha$ (solid) phase.
As the temperature changes, the relative vertical position of the liquid and solid G curve change. We can see that:
\begin{itemize}
    \item At T=1400°C the curves intersect near the Ni side and the common tangent points move to the right(See Fig. \ref{fig:dario1_comp})
    \item At T=1300°C, as temperature increases, the liquid phase becomes more stable relative to the solid and the common tangent shift to the center (higher Ni concentration).(See Fig. \ref{fig:dario2_comp})
    \item At T=1200°C the curves intersect at low Ni concentrations. (See Fig. \ref{fig:dario5_comp})
    \item At T=1100°C the solid curve is shifted significantly down compared to the liquid curve, and the two curves intersect only at a very low nickel concentration.(See Fig. \ref{fig:dario3_comp})
\end{itemize}    
So, by repeating this process at every temperature we can trace out two continuous lines, the liquidus line (formed by connecting all the $X^L$ points) and the solidus line (formed by connecting all the $X^{\alpha}$ points).

% \begin{figure}[H]
%     \centering
%     \includegraphics[width=\linewidth]{images/cuni1000.jpg}
%     \caption{Energy curve @ 1000°C}
%     \label{fig:dario4}
% \end{figure}

\newpage
% \paragraph{Exercise 3} Spiegare cosa sono le $\Omega$, commenti (Fra)

% \begin{figure}[H]
%     \centering
%     \includegraphics[width=\linewidth]{images/0_0.jpg}
%     \caption{$\Omega_l$ = 0, $\Omega_s$ = 0}
%     \label{fig:myimage1}
% \end{figure}
% \begin{figure}[H]
%     \centering
%     \includegraphics[width=\linewidth]{images/0_15.jpg}
%     \caption{$\Omega_l$ = 0, $\Omega_s$ = 15}
%     \label{fig:myimage2}
% \end{figure}
% \begin{figure}[H]
%     \centering
%     \includegraphics[width=\linewidth]{images/10_0.jpg}
%     \caption{$\Omega_l$ = 10, $\Omega_s$ = 0}
%     \label{fig:myimage3}
% \end{figure}
% \begin{figure}[H]
%     \centering
%     \includegraphics[width=\linewidth]{images/20_0.jpg}
%     \caption{$\Omega_l$ = 20, $\Omega_s$ = 0}
%     \label{fig:myimage4}
% \end{figure}

\subsection{Exercise 3} 
We calculated 4 phase diagrams corresponding to different combinations the values of $\Omega_l$ and $\Omega_s$ for the regular solution model.

\begin{figure}[H]
    \centering
    \includegraphics[width=0.75\linewidth]{images/0_0.jpg}
    \caption{$\Omega_l$ = 0, $\Omega_s$ = 0; Using a regular solution parameter = 0 means that here the model used is the ideal solution, hence, $\Delta H_{mix}$ here is null.}
    \label{fig:myimage1}
\end{figure}
\begin{figure}[H]
    \centering
    \includegraphics[width=0.75\linewidth]{images/0_15.jpg}
    \caption{$\Omega_l$ = 0, $\Omega_s$ = 15; In this case, the liquid phase behaves as an ideal solution. As a result, the liquid remains fully miscible over the entire composition range, and no liquid-liquid separation occurs.
    Conversely, the solid phase is characterized by a positive interaction parameter, which favours demixing between components A and B in the solid state. This energetic penalty overcomes the entropic contribution to mixing, leading to solid-state phase separation.
    Consequently, two distinct solid solutions, $\alpha$ (A-rich) and $\beta$ (B-rich), are stable at low temperatures. The result is an eutectic phase diagram.}
 
    \label{fig:myimage2}
\end{figure}
\begin{figure}[H]
    \centering
    \includegraphics[width=0.75\linewidth]{images/10_0.jpg}
    \caption{$\Omega_l$ = 10, $\Omega_s$ = 0; Here, the situation is reversed. The solid phase behaves as an ideal solution, hence there is no energetic driving force for solid-state demixing. As a result, separate $\alpha$ and $\beta$ solid phases are not stable, and only a single mixed solid solution exists over the entire composition range.
    The liquid phase, however, has a positive interaction parameter, indicating unfavorable A-B interactions. As a consequence,  the liquid shows a reduced stability range and strongly curved downward the liquidus line.    
    The solid remains fully miscible, forming only a single solid solution.}
    \label{fig:myimage3}
\end{figure}
\begin{figure}[H]
    \centering
    \includegraphics[width=0.75\linewidth]{images/20_0.jpg}
    \caption{$\Omega_l$ = 20, $\Omega_s$ = 0; increasing $\Omega_l$, the liquidus line is higher in temperatures and the regions liq+$\alpha$ and liq+$\beta$ are more extended}
    \label{fig:myimage4}
\end{figure}

\newpage
\subsection{Exercise 4}

\begin{figure}[H]
    \centering
    \includegraphics[width=\linewidth]{images/Ag-Cu Phase diagram.jpg}
    \caption{Phase diagram Ag-Cu}
    \label{fig:myimage5}
\end{figure}

Phase present are: Liquid , Cu , Ag.  Two-phase fields: Liquid+Cu, Liquid+Ag, Cu+Ag.
\paragraph{Invariant Points:}
\begin{itemize}
    \item Eutectic x(Ag) = 0.6 at T = 779 C Reaction: L→Cu+Ag. 
\end{itemize}

\noindent The difference between the two phase diagrams (Fig. \ref{fig:dario6} and Fig. \ref{fig:myimage5}) is that in the Cu-Ni we don't have a eutectic point and the three two-phase fields.
We compute the free energy curves at different constant temperatures as a function of x(Ag) and we confront them with the phase diagram.

% \begin{figure}[H]
%     \centering
%     \includegraphics[width=\linewidth]{images/Ag-Cu gibbs T 2000.jpg}
%     \caption{Energy curve @ 2000°C}
%     \label{fig:myimage6}
% \end{figure}

\begin{figure}[H]
    \centering
    \includegraphics[width=\linewidth]{images/Ag-Cu gibbs T 1100_comp.jpg}
    \caption{Energy curve @ 1100°C}
    \label{fig:myimage7}
\end{figure}
% \begin{figure}[H]
%     \centering
%     \includegraphics[width=\linewidth]{images/Ag-Cu gibbs T 1084.jpg}
%     \caption{Energy curve @ 1084°C}
%     \label{fig:myimage8}
% \end{figure}
% \begin{figure}[H]
%     \centering
%     \includegraphics[width=\linewidth]{images/Ag-Cu gibbs T 970.jpg}
%     \caption{Energy curve @ 970°C}
%     \label{fig:myimage9}
% \end{figure}

\noindent At x(Ag) = 0 (Fig. \ref{fig:myimage7}) we have the melting point of Cu.
In this case using the free energy curves we know that we have one-phase field, and the liquid is stable in any composition. We can say that at 1100 °C we have only the liquid phase.

% \begin{figure}[H]
%     \centering
%     \includegraphics[width=\linewidth]{images/Ag-Cu gibbs T 961.jpg}
%     \caption{Energy curve @ 961°C}
%     \label{fig:myimage10}
% \end{figure}
\begin{figure}[H]
    \centering
    \includegraphics[width=0.85\linewidth]{images/Ag-Cu gibbs T 800_comp.png}
    \caption{Energy curve @ 800°C}
    \label{fig:myimage11}
\end{figure}

At this temperature (Fig. \ref{fig:myimage11}) we observe:
\begin{itemize}
    \item Three single-phase fields that are: 
    \begin{itemize}
        \item From X(Ag) = 0 to X$^{Cu}$  and this is the Cu phase.
        \item From X$^{L}$ to X$^{L}$ and this is the Liquid phase.
        \item From X$^{Ag}$ to X(Ag) = 1 and this is the Ag phase.
    \end{itemize}
    \item Two-phase fields that are:
    \begin{itemize}
        \item From X$^{Cu}$ to X$^{L}$ and this is the Liquid + Cu phase.
        \item From X$^{L}$ to X$^{Ag}$ and this is the Liquid + Ag phase.
    \end{itemize}
    \item Two common tangent that evidence the Two-phase fields.
\end{itemize}

% \begin{figure}[H]
%     \centering
%     \includegraphics[width=\linewidth]{images/Ag-Cu gibbs T 779.jpg}
%     \caption{Energy curve @ 779°C}
%     \label{fig:myimage13}
% \end{figure}

\begin{figure}[H]
    \centering
    \includegraphics[width=\linewidth]{images/Ag-Cu gibbs T 700_comp.jpg}
    \caption{Energy curve @ 700°C}
    \label{fig:myimage16}
\end{figure}

\noindent At 700°C (Fig. \ref{fig:myimage16}) we have one common tangent.
We observe the presence of two-phase fields Cu + Ag (From X$^{Cu}$ to X$^{Ag}$) and two single-phase fields Cu (From x(Ag) = 0 to X$^{Cu}$) and Ag (From X$^{Ag}$ to x(Ag) = 1).

\begin{figure}[H]
    \centering
    \includegraphics[width=\linewidth]{images/Ag-Cu gibbs T 500_comp.jpg}
    \caption{Energy curve @ 500°C}
    \label{fig:myimage15}
\end{figure}

\noindent At 500°C (Fig. \ref{fig:myimage15}) the free energy of the liquid phase is increased, moving up and away from that of the solid one; the two-phase fields Cu + Ag are becoming larger.
Combining all the information obtained with the energy curves at different temperatures, it is possible to create the phase diagram.

% \begin{figure}[H]
%     \centering
%     \includegraphics[width=\linewidth]{images/Ag-Cu gibbs T 200.jpg}
%     \caption{Energy curve @ 200°C}
%     \label{fig:myimage14}
% \end{figure}
% \begin{figure}[H]
%     \centering
%     \includegraphics[width=\linewidth]{images/Ag-Cu gibbs T 0.jpg}
%     \caption{Energy curve @ 0°C}
%     \label{fig:myimage12}
% \end{figure}

% \newpage
% \paragraph{Exercise 5}

% \begin{figure}[H]
%     \centering
%     \includegraphics[width=\linewidth]{images/solidification_curves.jpg}
%     \caption{Solidification curves}
%     \label{fig:solidification}
% \end{figure}

\newpage
\subsection{Exercise 5} We carried out an equilibrium and Scheil solidification
calculations for a Cu-Sn alloy with composition 5 at $\%Sn$.

\begin{figure}[H]
    \centering
    \includegraphics[width=0.75\linewidth]{images/solidification_curves.jpg}
    \caption{The graph compares the solidification temperature ($T$) as a function of the fraction of solid ($f_s$) for a $Cu$-5at$\%Sn$ alloy using two distinct thermodynamic models. Both models coincide  between $f_s = 0$ and $0.2$, where the first solid particles of Fcc (copper-rich $\alpha$ phase) begin to form at approximately 1010°C. As solidification progresses, the curves diverge. Under Lever Rule conditions, solidification completes at roughly 800°C, while in the Scheil simulation, we assume that there is not diffusion in the solid, causing the remaining liquid to become increasingly enriched with Sn. This lowers the liquidus temperature more aggressively than the equilibrium model.
    In addition to this, we notice 2 phase transitions that are not visible in the lever rule curve, nor in the phase diagram at this specific composition (see figure \ref{fig:solidification2}). The final solidification of the remaining liquid in the Scheil model is nearly 140°C lower than predicted by the Lever Rule.}
    \label{fig:solidification}
\end{figure}

\begin{figure}[H]
    \centering
    \includegraphics[width=0.75\linewidth]{images/immagine_fra.jpg}
    \caption{The vertical line is the composition @ 0.05 of $Sn$ and the horizontal line cuts the range of temperatures between 600°C and 1100°C as in the solidification curve in figure \ref{fig:solidification}. The phase transformations seen are: $Liq \to \alpha-Cu + Liq \to \alpha-Cu$ where $\alpha-Cu$ is the phase with fcc structure.}
    \label{fig:solidification2}
\end{figure}

\newpage

\subsection{Exercise 6} 

\subsection*{Cu-Li phase diagram} 
\begin{figure}[H]
    \centering
    \includegraphics[width=\linewidth]{images/Ternary/Cu-Li.jpg}
    \caption{Cu-Li phase diagram}
    \label{fig:ternary1}
\end{figure}

\noindent The phases that are present are: Liquid , Cu + liquid, Cu , Cu + Li.

\noindent 

\noindent We have one invariant point that is a eutectic reaction at approximately x(Li) =1 and T=180 °C. L→Cu+Li. This is a 3-phase equilibrium and so one invariant point.


\subsection*{Cu-Mg phase diagram}
\begin{figure}[H]
    \centering
    \includegraphics[width=\linewidth]{images/Ternary/Cu-Mg.jpg}
    \caption{Cu-Mg phase diagram}
    \label{fig:ternary2}
\end{figure}

\noindent Phase present are: Liquid , Cu , Mg, $\alpha$ , $\beta$

\vspace{10pt}

\noindent Two-phase fields: L + Cu, Cu + $\alpha$, L + $\alpha$, $\alpha$ + $\beta$, L + $\beta$, $\beta$ + Mg

\vspace{10pt}

\paragraph{Invariant points:}
\begin{itemize}
    \item Eutectic x(Mg) = 0.2 at T = 725 C: L→Cu+ $\alpha$
    \item Eutectic x(Mg) = 0.59 at T = 548 C: L→ $\alpha$ + $\beta$
    \item Eutectic x(Mg) = 0.83 at T = 484 C: L→ $\beta$ + Mg
\end{itemize}


\subsection*{Li-Mg phase diagram}
\begin{figure}[H]
    \centering
    \includegraphics[width=\linewidth]{images/Ternary/Li-Mg.jpg}
    \caption{}
    \label{fig:ternary3}
\end{figure}

Phase present are: Liquid, $\alpha$, $\beta$

\vspace{10pt}

\noindent Two-phase fields: L + $\alpha$, L + $\beta$, $\alpha$ + $\beta$

\vspace{10pt}

\paragraph{Invariant points:}

\begin{itemize}
    \item Eutectic x(Mg) = 0.79 at T = 590 C: L→$\alpha$ + $\beta$
\end{itemize}


\subsection*{Phase diagram Li-Mg-Cu}
\begin{figure}[H]
    \centering
    \includegraphics[width=\linewidth]{images/Ternary/non so.jpg}
    \caption{}
    \label{fig:ternary4}
\end{figure}

The amount of each phase present at the composition Cu = 0.3, Li = 0.25 Mg = 0.45 is:

\begin{itemize}
    \item Fraction of BCC\_A2 is 0.39
    \item Fraction of CUMG2 is 0.31
    \item Fraction of LAVES\_C15 is 0.29
\end{itemize}


