\section{Exercises with PANDAT}

\subsection{Exercise 2 \textcolor{red}{Commenti Aleksej e Dario}}
\begin{figure}[H]
    \centering
    \includegraphics[width=\linewidth]{images/Cu-Ni Phase diagram.jpg}
    \caption{Phase diagram Cu-Ni}
    \label{fig:dario6}
\end{figure}
\begin{figure}[H]
    \centering
    \includegraphics[width=\linewidth]{images/Leva_edited.jpg}
    \caption{Lever rule on Cu-Ni}
    \label{fig:leva}
\end{figure}

Fraction of each phase: 
\begin{itemize}
    \item Liquid: $F_L = \frac{S}{R + S} = \frac{C_{\alpha} - C_0}{C_{\alpha} - C_L} = 0.333$
    \item $\alpha$: $F_{\alpha} = \frac{R}{R + S} = \frac{C_0 - C_L}{C_{\alpha} - C_L} = 0.667$
\end{itemize}

\noindent Using the point calculation we obtain:
\begin{itemize}
    \item Liquid: 0.38
    \item $\alpha$: 0.62
\end{itemize}



\begin{figure}[H]
    \centering
    \includegraphics[width=\linewidth]{images/cuni1500.jpg}
    \caption{Energy curve @ 1500°C}
    \label{fig:dario7}
\end{figure}
\begin{figure}[H]
    \centering
    \includegraphics[width=\linewidth]{images/cuni1400_comp.jpg}
    \caption{Tangent method @ 1400°C}
    \label{fig:dario1_comp}
\end{figure}
\begin{figure}[H]
    \centering
    \includegraphics[width=\linewidth]{images/cuni1300_comp.jpg}
    \caption{Tangent method @ 1300°C}
    \label{fig:dario2_comp}
\end{figure}
\begin{figure}[H]
    \centering
    \includegraphics[width=\linewidth]{images/cuni1200_comp.jpg}
    \caption{Tangent method @ 1200°C}
    \label{fig:dario5_comp}
\end{figure}
\begin{figure}[H]
    \centering
    \includegraphics[width=\linewidth]{images/cuni1100_comp.jpg}
    \caption{Tangent method @ 1100°C}
    \label{fig:dario3_comp}
\end{figure}
\begin{figure}[H]
    \centering
    \includegraphics[width=\linewidth]{images/cuni1000.jpg}
    \caption{Energy curve @ 1000°C}
    \label{fig:dario4}
\end{figure}

\newpage
% \paragraph{Exercise 3} Spiegare cosa sono le $\Omega$, commenti (Fra)

% \begin{figure}[H]
%     \centering
%     \includegraphics[width=\linewidth]{images/0_0.jpg}
%     \caption{$\Omega_l$ = 0, $\Omega_s$ = 0}
%     \label{fig:myimage1}
% \end{figure}
% \begin{figure}[H]
%     \centering
%     \includegraphics[width=\linewidth]{images/0_15.jpg}
%     \caption{$\Omega_l$ = 0, $\Omega_s$ = 15}
%     \label{fig:myimage2}
% \end{figure}
% \begin{figure}[H]
%     \centering
%     \includegraphics[width=\linewidth]{images/10_0.jpg}
%     \caption{$\Omega_l$ = 10, $\Omega_s$ = 0}
%     \label{fig:myimage3}
% \end{figure}
% \begin{figure}[H]
%     \centering
%     \includegraphics[width=\linewidth]{images/20_0.jpg}
%     \caption{$\Omega_l$ = 20, $\Omega_s$ = 0}
%     \label{fig:myimage4}
% \end{figure}

\subsection{Exercise 3} 
We calculated 4 phase diagrams corresponding to different combinations the values of $\Omega_l$ and $\Omega_s$ for the regular solution model.

\begin{figure}[H]
    \centering
    \includegraphics[width=0.75\linewidth]{images/0_0.jpg}
    \caption{$\Omega_l$ = 0, $\Omega_s$ = 0; Using a regular solution parameter = 0 means that here, the model used is the ideal solution, hence, $\Delta H_{mix}$ here is null.}
    \label{fig:myimage1}
\end{figure}
\begin{figure}[H]
    \centering
    \includegraphics[width=0.75\linewidth]{images/0_15.jpg}
    \caption{$\Omega_l$ = 0, $\Omega_s$ = 15; In this case, the liquid phase behaves as an ideal solution. As a result, the liquid remains fully miscible over the entire composition range, and no liquid-liquid separation occurs.
    Conversely, the solid phase is characterized by a positive interaction parameter, which favours demixing between components A and B in the solid state. This energetic penalty overcomes the entropic contribution to mixing, leading to solid-state phase separation.
    Consequently, two distinct solid solutions, $\alpha$ (A-rich) and $\beta$ (B-rich), are stable at low temperatures. The result is an eutectic phase diagram.}
 
    \label{fig:myimage2}
\end{figure}
\begin{figure}[H]
    \centering
    \includegraphics[width=0.75\linewidth]{images/10_0.jpg}
    \caption{$\Omega_l$ = 10, $\Omega_s$ = 0; Here, the situation is reversed. The solid phase behaves as an ideal solution, hence there is no energetic driving force for solid-state demixing. As a result, separate $\alpha$ and $\beta$ solid phases are not stable, and only a single mixed solid solution exists over the entire composition range.
    The liquid phase, however, has a positive interaction parameter, indicating unfavorable A-B interactions. As a consequence,  the liquid shows a reduced stability range and strongly curved downward the liquidus line.    
    The solid remains fully miscible, forming only a single solid solution.}
    \label{fig:myimage3}
\end{figure}
\begin{figure}[H]
    \centering
    \includegraphics[width=0.75\linewidth]{images/20_0.jpg}
    \caption{$\Omega_l$ = 20, $\Omega_s$ = 0; increasing $\Omega_l$, the liquidus line is higher in temperatures and the regions liq+$\alpha$ and liq+$\beta$ are more extended}
    \label{fig:myimage4}
\end{figure}

\newpage
\subsection{Exercise 4} \textcolor{red}{Commenti (Aleksej e Dario). Metodo della tangente (Ale).}

\begin{figure}[H]
    \centering
    \includegraphics[width=\linewidth]{images/Ag-Cu Phase diagram.jpg}
    \caption{Phase diagram Ag-Cu}
    \label{fig:myimage5}
\end{figure}

\begin{figure}[H]
    \centering
    \includegraphics[width=\linewidth]{images/Ag-Cu gibbs T 2000.jpg}
    \caption{Energy curve @ 2000°C}
    \label{fig:myimage6}
\end{figure}

\begin{figure}[H]
    \centering
    \includegraphics[width=\linewidth]{images/Ag-Cu gibbs T 1100_comp.jpg}
    \caption{Energy curve @ 1100°C. The free energy curve of the liquid lies below that of the solid throughout the entire composition range; therefore, the only stable phase is the liquid.}
    \label{fig:myimage7}
\end{figure}
\begin{figure}[H]
    \centering
    \includegraphics[width=\linewidth]{images/Ag-Cu gibbs T 1084.jpg}
    \caption{Energy curve @ 1084°C}
    \label{fig:myimage8}
\end{figure}
\begin{figure}[H]
    \centering
    \includegraphics[width=\linewidth]{images/Ag-Cu gibbs T 970.jpg}
    \caption{Energy curve @ 970°C}
    \label{fig:myimage9}
\end{figure}
\begin{figure}[H]
    \centering
    \includegraphics[width=\linewidth]{images/Ag-Cu gibbs T 961.jpg}
    \caption{Energy curve @ 961°C}
    \label{fig:myimage10}
\end{figure}
\begin{figure}[H]
    \centering
    \includegraphics[width=\linewidth]{images/Ag-Cu gibbs T 800_comp.png}
    \caption{Energy curve @ 800°C}
    \label{fig:myimage11}
\end{figure}
\begin{figure}[H]
    \centering
    \includegraphics[width=\linewidth]{images/Ag-Cu gibbs T 779.jpg}
    \caption{Energy curve @ 779°C}
    \label{fig:myimage13}
\end{figure}

\begin{figure}[H]
    \centering
    \includegraphics[width=\linewidth]{images/Ag-Cu gibbs T 700_comp.jpg}
    \caption{Energy curve @ 700°C}
    \label{fig:myimage16}
\end{figure}

\begin{figure}[H]
    \centering
    \includegraphics[width=\linewidth]{images/Ag-Cu gibbs T 500_comp.jpg}
    \caption{Energy curve @ 500°C}
    \label{fig:myimage15}
\end{figure}

\begin{figure}[H]
    \centering
    \includegraphics[width=\linewidth]{images/Ag-Cu gibbs T 200.jpg}
    \caption{Energy curve @ 200°C}
    \label{fig:myimage14}
\end{figure}
\begin{figure}[H]
    \centering
    \includegraphics[width=\linewidth]{images/Ag-Cu gibbs T 0.jpg}
    \caption{Energy curve @ 0°C}
    \label{fig:myimage12}
\end{figure}

% \newpage
% \paragraph{Exercise 5} \textcolor{red}{Commento Cu-Sn (Fra)}

% \begin{figure}[H]
%     \centering
%     \includegraphics[width=\linewidth]{images/solidification_curves.jpg}
%     \caption{Solidification curves}
%     \label{fig:solidification}
% \end{figure}

\newpage
\subsection{Exercise 5} We carried out an equilibrium and Scheil solidification
calculations for a Cu-Sn alloy with composition 5at$\%Sn$.

\begin{figure}[H]
    \centering
    \includegraphics[width=\linewidth]{images/solidification_curves.jpg}
    \caption{The graph compares the solidification temperature ($T$) as a function of the fraction of solid ($f_s$) for a Cu-5at$\%Sn$ alloy using two distinct thermodynamic models. Both models coincide  between $f_s = 0$ and $0.2$, where the first solid particles of Fcc (copper-rich $\alpha$ phase) begin to form at approximately 1010°C. As solidification progresses, the curves diverge. Under Lever Rule conditions,solidification completes at roughly 800°C, while in the Scheil simulation, we assume that there is not diffusion in the solid, causing the remaining liquid to become increasingly enriched with Sn. This lowers the liquidus temperature more aggressively than the equilibrium model.
    In addition to this, we notice 2 phase transitions that are not visible in the lever rule curve, nor in the phase diagram at this specific composition (see figure \ref{fig:solidification2}). The final solidification of the remaining liquid in the Scheil model is nearly 140°C lower than predicted by the Lever Rule.}
    \label{fig:solidification}
\end{figure}

\begin{figure}[H]
    \centering
    \includegraphics[width=\linewidth]{images/immagine_fra.jpg}
    \caption{The vertical line is the composition @ 0.05 of $Cu$ and the horizontal line cuts the range of temperatures between 600°C and 1100°C as in the solidification curve in figure \ref{fig:solidification}}
    \label{fig:solidification2}
\end{figure}

\subsection{Exercise 6} 

\subsection*{Cu-Li phase diagram} 
\begin{figure}[H]
    \centering
    \includegraphics[width=\linewidth]{images/Ternary/Cu-Li.jpg}
    \caption{}
    \label{fig:ternary1}
\end{figure}

\noindent The phases that are present are: Liquid , Cu + liquid, Cu , Cu + Li.

\noindent 

\noindent We have one invariant point that is a eutectic reaction at approximately x(Li) =1 and T=180 °C. L→Cu+Li. This is a 3-phase equilibrium and so one invariant point.


\subsection*{Cu-Mg phase diagram}
\begin{figure}[H]
    \centering
    \includegraphics[width=\linewidth]{images/Ternary/Cu-Mg.jpg}
    \caption{}
    \label{fig:ternary2}
\end{figure}

\noindent Phase present are: Liquid , Cu , Mg, $\alpha$ , $\beta$

\vspace{10pt}

\noindent Two-phase fields: L + Cu, Cu + $\alpha$, L + $\alpha$, $\alpha$ + $\beta$, L + $\beta$, $\beta$ + Mg

\vspace{10pt}

\noindent Invariant points:
\begin{itemize}
    \item Eutectic x(Mg) = 0.2 at T = 725 C: L→Cu+ $\alpha$
    \item Eutectic x(Mg) = 0.59 at T = 548 C: L→ $\alpha$ + $\beta$
    \item Eutectic x(Mg) = 0.83 at T = 484 C: L→ $\beta$ + Mg
\end{itemize}


\subsection*{Li-Mg phase diagram}
\begin{figure}[H]
    \centering
    \includegraphics[width=\linewidth]{images/Ternary/Li-Mg.jpg}
    \caption{}
    \label{fig:ternary3}
\end{figure}

Phase present are: Liquid, $\alpha$, $\beta$

\vspace{10pt}

\noindent Two-phase fields: L + $\alpha$, L + $\beta$, $\alpha$ + $\beta$

\vspace{10pt}

\noindent Invariant points:

\begin{itemize}
    \item Eutectic x(Mg) = 0.79 at T = 590 C: L→$\alpha$ + $\beta$
\end{itemize}


\subsection*{Phase diagram Li-Mg-Cu}
\begin{figure}[H]
    \centering
    \includegraphics[width=\linewidth]{images/Ternary/non so.jpg}
    \caption{}
    \label{fig:ternary4}
\end{figure}

The amount of each phase present at the composition Cu = 0.3, Li = 0.25 Mg = 0.45 is:

\begin{itemize}
    \item Fraction of BCC\_A2 is 0.39
    \item Fraction of CUMG2 is 0.31
    \item Fraction of LAVES\_C15 is 0.29
\end{itemize}


