\section{Exercises with PANDAT}

\paragraph{Exercise 1}
Exercise done during the class-explanation.

\newpage
\paragraph{Exercise 2} \textcolor{red}{Manca il grafico della leva (graficamente e con PANDAT) solo su un grafico (Dario). Per tutti i grafici mancanti riportare le linee di proiezione con InkScape (Ale). Commenti (Fra).
}
\begin{figure}[H]
    \centering
    \includegraphics[width=\linewidth]{images/Cu-Ni Phase diagram.jpg}
    \caption{Phase diagram Cu-Ni}
    \label{fig:dario6}
\end{figure}
\begin{figure}[H]
    \centering
    \includegraphics[width=\linewidth]{images/cuni1500.jpg}
    \caption{Energy curve @ 1500°C}
    \label{fig:dario7}
\end{figure}
\begin{figure}[H]
    \centering
    \includegraphics[width=\linewidth]{images/cuni1400.jpg}
    \caption{Energy curve @ 1400°C}
    \label{fig:dario1}
\end{figure}
\begin{figure}[H]
    \centering
    \includegraphics[width=\linewidth]{images/cuni1400_comp.jpg}
    \caption{Tangent method @ 1400°C}
    \label{fig:dario1_comp}
\end{figure}
\begin{figure}[H]
    \centering
    \includegraphics[width=\linewidth]{images/cuni1300.jpg}
    \caption{Energy curve @ 1300°C}
    \label{fig:dario2}
\end{figure}
\begin{figure}[H]
    \centering
    \includegraphics[width=\linewidth]{images/cuni1300_comp.jpg}
    \caption{Tangent method @ 1300°C}
    \label{fig:dario2_comp}
\end{figure}
\begin{figure}[H]
    \centering
    \includegraphics[width=\linewidth]{images/cuni1200.jpg}
    \caption{Energy curve @ 1200°C}
    \label{fig:dario5}
\end{figure}
\begin{figure}[H]
    \centering
    \includegraphics[width=\linewidth]{images/cuni1200_comp.jpg}
    \caption{Tangent method @ 1200°C}
    \label{fig:dario5_comp}
\end{figure}
\begin{figure}[H]
    \centering
    \includegraphics[width=\linewidth]{images/cuni1100.jpg}
    \caption{Energy curve @ 1100°C}
    \label{fig:dario3}
\end{figure}
\begin{figure}[H]
    \centering
    \includegraphics[width=\linewidth]{images/cuni1100_comp.jpg}
    \caption{Tangent method @ 1100°C}
    \label{fig:dario3_comp}
\end{figure}
\begin{figure}[H]
    \centering
    \includegraphics[width=\linewidth]{images/cuni1000.jpg}
    \caption{Energy curve @ 1000°C}
    \label{fig:dario4}
\end{figure}

\newpage
% \paragraph{Exercise 3} Spiegare cosa sono le $\Omega$, commenti (Fra)

% \begin{figure}[H]
%     \centering
%     \includegraphics[width=\linewidth]{images/0_0.jpg}
%     \caption{$\Omega_l$ = 0, $\Omega_s$ = 0}
%     \label{fig:myimage1}
% \end{figure}
% \begin{figure}[H]
%     \centering
%     \includegraphics[width=\linewidth]{images/0_15.jpg}
%     \caption{$\Omega_l$ = 0, $\Omega_s$ = 15}
%     \label{fig:myimage2}
% \end{figure}
% \begin{figure}[H]
%     \centering
%     \includegraphics[width=\linewidth]{images/10_0.jpg}
%     \caption{$\Omega_l$ = 10, $\Omega_s$ = 0}
%     \label{fig:myimage3}
% \end{figure}
% \begin{figure}[H]
%     \centering
%     \includegraphics[width=\linewidth]{images/20_0.jpg}
%     \caption{$\Omega_l$ = 20, $\Omega_s$ = 0}
%     \label{fig:myimage4}
% \end{figure}

\paragraph{Exercise 3} 
We calculated 4 phase diagrams corresponding to different combinations the values of $\Omega_l$ and $\Omega_s$ for the regular solution model.

\begin{figure}[H]
    \centering
    \includegraphics[width=\linewidth]{images/0_0.jpg}
    \caption{$\Omega_l$ = 0, $\Omega_s$ = 0; Using a regular solution parameter = 0 means that here, the model used is the ideal solution, hence, $\Delta H_{mix}$ here is null.}
    \label{fig:myimage1}
\end{figure}
\begin{figure}[H]
    \centering
    \includegraphics[width=\linewidth]{images/0_15.jpg}
    \caption{$\Omega_l$ = 0, $\Omega_s$ = 15; In this case, the liquid phase behaves as an ideal solution. As a result, the liquid remains fully miscible over the entire composition range, and no liquid-liquid separation occurs.
    Conversely, the solid phase is characterized by a positive interaction parameter, which favours demixing between components A and B in the solid state. This energetic penalty overcomes the entropic contribution to mixing, leading to solid-state phase separation.
    Consequently, two distinct solid solutions, $\alpha$ (A-rich) and $\beta$ (B-rich), are stable at low temperatures. The result is an eutectic phase diagram.}
 
    \label{fig:myimage2}
\end{figure}
\begin{figure}[H]
    \centering
    \includegraphics[width=\linewidth]{images/10_0.jpg}
    \caption{$\Omega_l$ = 10, $\Omega_s$ = 0; Here, the situation is reversed. The solid phase behaves as an ideal solution, hence there is no energetic driving force for solid-state demixing. As a result, separate $\alpha$ and $\beta$ solid phases are not stable, and only a single mixed solid solution exists over the entire composition range.
    The liquid phase, however, has a positive interaction parameter, indicating unfavorable A-B interactions. As a consequence,  the liquid shows a reduced stability range and strongly curved downward the liquidus line.    
    The solid remains fully miscible, forming only a single solid solution.}
    \label{fig:myimage3}
\end{figure}
\begin{figure}[H]
    \centering
    \includegraphics[width=\linewidth]{images/20_0.jpg}
    \caption{$\Omega_l$ = 20, $\Omega_s$ = 0; increasinig $\Omega_l$, the liquidus line is higher in temperatures and the regions liq+$\alpha$ and liq+$\beta$ are more extended}
    \label{fig:myimage4}
\end{figure}

\newpage
\paragraph{Exercise 4} \textcolor{red}{Manca il grafico della leva (graficamente e con PANDAT) solo su un grafico (Dario). Per tutti i grafici mancanti riportare le linee di proiezione con InkScape (Ale). Commenti (Fra).}

\begin{figure}[H]
    \centering
    \includegraphics[width=\linewidth]{images/Ag-Cu Phase diagram.jpg}
    \caption{Phase diagram Ag-Cu}
    \label{fig:myimage5}
\end{figure}

\begin{figure}[H]
    \centering
    \includegraphics[width=\linewidth]{images/Ag-Cu gibbs T 2000.jpg}
    \caption{Energy curve @ 2000°C}
    \label{fig:myimage6}
\end{figure}

\begin{figure}[H]
    \centering
    \includegraphics[width=\linewidth]{images/Ag-Cu gibbs T 1100.jpg}
    \caption{Energy curve @ 1100°C}
    \label{fig:myimage7}
\end{figure}
\begin{figure}[H]
    \centering
    \includegraphics[width=\linewidth]{images/Ag-Cu gibbs T 1084.jpg}
    \caption{Energy curve @ 1084°C}
    \label{fig:myimage8}
\end{figure}
\begin{figure}[H]
    \centering
    \includegraphics[width=\linewidth]{images/Ag-Cu gibbs T 970.jpg}
    \caption{Energy curve @ 970°C}
    \label{fig:myimage9}
\end{figure}
\begin{figure}[H]
    \centering
    \includegraphics[width=\linewidth]{images/Ag-Cu gibbs T 961.jpg}
    \caption{Energy curve @ 961°C}
    \label{fig:myimage10}
\end{figure}
\begin{figure}[H]
    \centering
    \includegraphics[width=\linewidth]{images/Ag-Cu gibbs T 800.jpg}
    \caption{Energy curve @ 800°C}
    \label{fig:myimage11}
\end{figure}
\begin{figure}[H]
    \centering
    \includegraphics[width=\linewidth]{images/Ag-Cu gibbs T 779.jpg}
    \caption{Energy curve @ 779°C}
    \label{fig:myimage13}
\end{figure}

\begin{figure}[H]
    \centering
    \includegraphics[width=\linewidth]{images/Ag-Cu gibbs T 700.jpg}
    \caption{Energy curve @ 700°C}
    \label{fig:myimage16}
\end{figure}

\begin{figure}[H]
    \centering
    \includegraphics[width=\linewidth]{images/Ag-Cu gibbs T 500.jpg}
    \caption{Energy curve @ 500°C}
    \label{fig:myimage15}
\end{figure}

\begin{figure}[H]
    \centering
    \includegraphics[width=\linewidth]{images/Ag-Cu gibbs T 200.jpg}
    \caption{Energy curve @ 200°C}
    \label{fig:myimage14}
\end{figure}
\begin{figure}[H]
    \centering
    \includegraphics[width=\linewidth]{images/Ag-Cu gibbs T 0.jpg}
    \caption{Energy curve @ 0°C}
    \label{fig:myimage12}
\end{figure}

% \newpage
% \paragraph{Exercise 5} \textcolor{red}{Commento Cu-Sn (Fra)}

% \begin{figure}[H]
%     \centering
%     \includegraphics[width=\linewidth]{images/solidification_curves.jpg}
%     \caption{Solidification curves}
%     \label{fig:solidification}
% \end{figure}

\newpage
\paragraph{Exercise 5} We carried out an equilibrium and Scheil solidification
calculations for a Cu-Sn alloy with composition 5at$\%Sn$.

\begin{figure}[H]
    \centering
    \includegraphics[width=\linewidth]{images/solidification_curves.jpg}
    \caption{The graph compares the solidification temperature ($T$) as a function of the fraction of solid ($f_s$) for a Cu-5at$\%Sn$ alloy using two distinct thermodynamic models. Both models coincide  between $f_s = 0$ and $0.2$, where the first solid particles of Fcc (copper-rich $\alpha$ phase) begin to form at approximately 1010°C. As solidification progresses, the curves diverge. Under Lever Rule conditions,solidification completes at roughly 800°C, while in the Scheil simulation, we assume that there is not diffusion in the solid, causing the remaining liquid to become increasingly enriched with Sn. This lowers the liquidus temperature more aggressively than the equilibrium model.
    In addition to this, we notice 2 phase transitions that are not visible in the lever rule curve, nor in the phase diagram at this specific composition (see next figure). The final solidification of the remaining liquid in the Scheil model is nearly 140°C lower than predicted by the Lever Rule.}
    \label{fig:solidification}
\end{figure}

\begin{figure}[H]
    \centering
    \includegraphics[width=\linewidth]{images/immagine_fra.jpg}
    \caption{\textcolor{red}{Fra fai commento}}
    \label{fig:solidification2}
\end{figure}

\newpage
\paragraph{Exercise 6} \textcolor{red}{Diagramma di fase a 3 componenti (Aleksej - Fra)}
