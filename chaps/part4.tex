\newpage
\section{Exercise with FreeFem++}

\subsection{Exercise 1}
Deformation of an Elliptical Membrane with Spatially Varying Stiffness

\paragraph{Case 1}: 
\begin{lstlisting}[style=cppstyle]
// ex 1
load "medit"

bool debug=true;

real a = 2;
real b = 3;

// Define mesh boundary
border C(t=0, 2*pi){x= a*cos(t); y=b*sin(t);}

// The triangulated domain Th is on the left side of its boundary
mesh Th = buildmesh(C(1000));
plot(Th, wait=debug);

// The finite element space defined over Th is called here Vh 
fespace Vh(Th, P2);
Vh u, v;// Define u and v as piecewise-P1 continuous functions

// Define a function f for the stiffness

real k0 = 1;
real alpha = 0.5;

func k = k0*(1 + alpha*x);

// Get the clock in second
real cpu=clock();
macro grad(u) [dx(u), dy(u)] //

// Define the PDE
solve Poisson(u, v, solver=CG)
= int2d(Th)(grad(u)'*grad(v)*k)  // The bilinear part
- int2d(Th)(k*v)                         // The linear part 
+ on(C, u=0);                            // The Dirichlet boundary condition

// Plot the result
medit("u", Th, u, wait=debug);
\end{lstlisting}

\begin{figure}[H]
    \centering
    \includegraphics[width=0.5\linewidth]{images/fra_ex_1_1.png}
    \caption{Case 1: Smooth function of position}
    \label{fig:fra_ex_1_1}
\end{figure}

\newpage
\paragraph{Case 2}:
\begin{lstlisting}[style=cppstyle]
// ex 1
load "medit"

bool debug=true;

// external border
real a = 4;
real b = 6;

// Define mesh boundary
border C(t=0, 2*pi){x= a*cos(t); y=b*sin(t);}

// internal border
real aC = 2;
real bC = 3;

// The triangulated domain Th is on the left side of its boundary
mesh Th = buildmesh(C(1000));
plot(Th, wait=debug);

// The finite element space defined over Th is called here Vh 
fespace Vh(Th, P2);
Vh u, v;// Define u and v as piecewise-P2 continuous functions

// Define a function f for the stiffness

real ksoft = 0.5;
// real kstiff = 1; // ---> case 1
// real kstiff = 10; // ---> case 2
real kstiff = 100; // ---> case 3


func k = ksoft + (kstiff-ksoft)*((x*x)/(aC*aC)+(y*y)/(bC*bC) <= 1);

// Get the clock in second
real cpu=clock();
macro grad(u) [dx(u), dy(u)] //

// Define the PDE
solve Poisson(u, v, solver=CG)
= int2d(Th)(grad(u)'*grad(v)*k)  // The bilinear part
- int2d(Th)(k*v)                         // The linear part 
+ on(C, u=0);                            // The Dirichlet boundary condition

// Plot the result
medit("u", Th, u, wait=debug);
\end{lstlisting}

\begin{figure}[H]
    \centering
    \begin{minipage}{0.32\linewidth}
        \centering
        \includegraphics[width=\linewidth]{images/fra_ex_1_2_K1.png}
        \caption*{K = 1}
    \end{minipage}
    \hfill
    \begin{minipage}{0.32\linewidth}
        \centering
        \includegraphics[width=\linewidth]{images/fra_ex_1_2_K10.png}
        \caption*{K = 10}
    \end{minipage}
    \hfill
    \begin{minipage}{0.32\linewidth}
        \centering
        \includegraphics[width=\linewidth]{images/fra_ex_1_2_K1000.png}
        \caption*{K = 1000}
    \end{minipage}
    \caption{Case 2: Piecewise constant stiffness (material zones)}
    \label{fig:fra_ex_1_2}
\end{figure}

\paragraph{\textcolor{red}{Comments case 1}}
\paragraph{\textcolor{red}{Comments case 2}} If K increases, the gradient in the plot is more marked.

\newpage
\subsection{\textcolor{red}{Exercise 2}}
