\newpage
\section{Exercise with FreeFem++}

\subsection{Exercise 1}
Deformation of an Elliptical Membrane with Spatially Varying Stiffness.

\paragraph{Case 1}: 
\begin{lstlisting}[style=cppstyle]
// ex 1
load "medit"

bool debug=true;

real a = 2;
real b = 3;

// Define mesh boundary
border C(t=0, 2*pi){x= a*cos(t); y=b*sin(t);}

// The triangulated domain Th is on the left side of its boundary
mesh Th = buildmesh(C(1000));
plot(Th, wait=debug);

// The finite element space defined over Th is called here Vh 
fespace Vh(Th, P2);
Vh u, v;// Define u and v as piecewise-P2 continuous functions

// Define a function f for the stiffness

real k0 = 1;
real alpha = 0.5;

func k = k0*(1 + alpha*x);

// Get the clock in second
real cpu=clock();
macro grad(u) [dx(u), dy(u)] //

// Define the PDE
solve Poisson(u, v, solver=CG)
= int2d(Th)(grad(u)'*grad(v)*k)     // The bilinear part
- int2d(Th)(k*v)                    // The linear part 
+ on(C, u=0);                       // The Dirichlet boundary condition

// Plot the result
medit("u", Th, u, wait=debug);
\end{lstlisting}

\newpage
\paragraph{Case 2}:
\begin{lstlisting}[style=cppstyle]
// ex 1
load "medit"

bool debug=true;

// external border
real a = 4;
real b = 6;

// Define mesh boundary
border C(t=0, 2*pi){x= a*cos(t); y=b*sin(t);}

// internal border
real aC = 2;
real bC = 3;

// The triangulated domain Th is on the left side of its boundary
mesh Th = buildmesh(C(1000));
plot(Th, wait=debug);

// The finite element space defined over Th is called here Vh 
fespace Vh(Th, P2);
Vh u, v;// Define u and v as piecewise-P2 continuous functions
    
// Define a function f for the stiffness

real ksoft = 0.5;
// real kstiff = 1; // ---> case 1
// real kstiff = 10; // ---> case 2
real kstiff = 100; // ---> case 3


func k = ksoft + (kstiff-ksoft)*((x*x)/(aC*aC)+(y*y)/(bC*bC) <= 1);

// Get the clock in second
real cpu=clock();
macro grad(u) [dx(u), dy(u)] //

// Define the PDE
solve Poisson(u, v, solver=CG)
= int2d(Th)(grad(u)'*grad(v)*k)  // The bilinear part
- int2d(Th)(k*v)                         // The linear part 
+ on(C, u=0);                            // The Dirichlet boundary condition

// Plot the result
medit("u", Th, u, wait=debug);
\end{lstlisting}

\begin{figure}[H]
    \centering
    \includegraphics[width=0.5\linewidth]{images/fra_ex_1_1.png}
    \caption{Case 1: Smooth function of position}
    \label{fig:fra_ex_1_1}
\end{figure}

\begin{figure}[H]
    \centering
    \begin{minipage}{0.32\linewidth}
        \centering
        \includegraphics[width=\linewidth]{images/fra_ex_1_2_K1.png}
        \caption*{$k_{\mathrm{stiff}}$ = 1}
    \end{minipage}
    \hfill
    \begin{minipage}{0.32\linewidth}
        \centering
        \includegraphics[width=\linewidth]{images/fra_ex_1_2_K10.png}
        \caption*{$k_{\mathrm{stiff}}$ = 10}
    \end{minipage}
    \hfill
    \begin{minipage}{0.32\linewidth}
        \centering
        \includegraphics[width=\linewidth]{images/fra_ex_1_2_K1000.png}
        \caption*{$k_{\mathrm{stiff}}$ = 100}
    \end{minipage}
    \caption{Case 2: Piecewise constant stiffness (material zones)}
    \label{fig:fra_ex_1_2}
\end{figure}

\paragraph{Comments:}

Using a single $k_0$, the membrane deformation is gradual, with larger displacements occurring toward the center, as expected.  
When employing two coefficients, $k_{\mathrm{soft}}$ and $k_{\mathrm{stiff}}$, the deformation gradient becomes significantly more pronounced, particularly as $k_{\mathrm{stiff}}$ increases.


\newpage
\subsection{3D HYDROGEN DIFFUSION IN A BOLT}
This script simulates the diffusion of hydrogen in a 3D bolt using the Backward Euler time-stepping scheme.

\begin{lstlisting}[style=cppstyle]
// Load the library required to export results for Paraview/PyVista
load "iovtk"

// 1. MESH LOADING
mesh3 Th("vite_ale_refined.mesh"); 

// Bolt height: 4.05 (From gmsh project)
// Bolt width: 2 (From gmsh project)
// We assume all units of measure are in `cm'

// 2. PHYSICAL PARAMETERS
real cs = 10e-5;      // Surface concentration (exercise constraint)
real l = 0.1;         // Smoothness parameter for the transition (exercise constraint)
real dhead = 1.6e-6;  // Diffusion coefficient in the bolt head in cm^2/s (Assumed values)
real dshank = 0.8e-6; // Diffusion coefficient in the shank in cm^2/s (Assumed values)
real z0 = 0.5;        // Vertical position (z) of the head-shank interface

// Define the Diffusion Coefficient as a function of space D(z).
// We use a hyperbolic tangent `s' to create a smooth transition 
// between dhead and dshank to avoid numerical instability.
func s = 0.5 * (1 + tanh((z - z0) / l));
func D = dhead + (dshank - dhead) * s;

// 3. FINITE ELEMENT SPACE
// We define a P1 space (piecewise linear functions) on the mesh.
// c: current concentration, v: test function, cold: previous time step
fespace Vh(Th, P1);   
Vh c, v, cold;

// 4. TIME PARAMETERS
real dt = 3600;          // Time step size [seconds]
real Tmax = 3600 * 10;        // Final simulation time (10 hours)
real tCurr = 0.0;       // Current time (renamed from 'time' to avoid warnings)
int step = 0;

// 5. INITIAL CONDITION
// At t = 0, the bolt has zero hydrogen concentration.
c = 0;                  

// 6. WEAK FORMULATION (Backward Euler)
// The equation discretized is: (c^n - c^{n-1})/dt - div(D * grad(c^n)) = 0
// Weak form: $\int$ (c/dt)*v - $\int$ (cold/dt)*v + $\int$ D*grad(c) dot grad(v) = 0
problem transport(c, v)
    = int3d(Th)( (c / dt) * v )                
    - int3d(Th)( (cold / dt) * v )             
    + int3d(Th)( D * (dx(c)*dx(v) + dy(c)*dy(v) + dz(c)*dz(v)) )
    // Dirichlet Condition: c = cs on the exposed head (label = 2)
    + on(2, c = cs); 
    // The Neumann Condition (-D grad c dot n = 0) for the protected shank 
    // is the "natural" BC and is automatically satisfied by the weak form.

// 7. TIME ITERATIONS LOOP
int[int] Order = [1];        // VTK export parameter
string DataName = "concentration";

for (tCurr = 0; tCurr <= Tmax; tCurr += dt) {
    // Step A: Store the current solution as 'cold' before solving for the next step
    cold = c; 
    
    // Step B: Solve the linear system defined in the 'transport' problem
    transport; 

    // Step C: Save the result to a .vtk file for 3D visualization in PyVista
    // Each file is named with the current time (e.g., bolt_hydrogen0.1.vtk)
    savevtk("bolt_hydrogen_" + step + ".vtk", Th, c, dataname=DataNae, order=Ordpng;

    step++;
}

Vh zCoord = z; 
\end{lstlisting}


\begin{lstlisting}[style=pythonstyle]
import pyvista as pv
import glob
import os
import re

edpfile="ex_2_fra_corretto.edp"
command="FreeFem++ "
string=command+edpfile
print(f"Esecuzione di: {string}")
os.system(string)
\end{lstlisting}

\begin{lstlisting}[style=pythonstyle]
# File selection and sorting
# Natural sort (1, 2, 10 instead of 1, 10, 2)
def natural_sort(l): 
    return sorted(l, key=lambda x: int(re.findall(r'\d+', x)[0]) if re.findall(r'\d+', x) else 0)

vtk_files = natural_sort(glob.glb("bolt_hydrogen_*.vtpng)

if not vtk_files:
    raise RuntimeError("No fil bolt_hydrogen_*.vtk founpng)
print(f"Found {len(vtk_files)} VTK files")
\end{lstlisting}

\begin{lstlisting}[style=pythonstyle]
zs = [0.08, 0.12, 0.25, 3]  # slices inside the head and the shank
cs_value = 10e-5            # as in the .edp file 

for i in range(0, len(vtk_files)):
    fname = vtk_files[i]
    print(f"Plot timestep {i}: {fname}")

    mesh = pv.read(fname)

    for z in zs:
        # Mesh slicing
        slc = mesh.slice(normal='z', origin=(0, 0, z))
        
        # If slice intersects the mesh: plot
        if slc.n_points > 0:
            p = pv.Plotter()
            # ADDED: force custom range
            p.add_mesh(slc, 
                       scalars="concentration", 
                       cmap="turbo", 
                       clim=[0, cs_value]) 
            
            p.add_text(f"File: {fname}, z = {z}", font_size=10)
            p.view_xy() # Frontal 2D view
            p.screenshot(f"{fname}_{z}.png", transparent_background= True) 
\end{lstlisting}

\newpage
\subsection{Image dump}

We selected a temporal range of 10 hours. Each image along a row corresponds to a cross-section of the screw within the head and shank.  

The time-sequence images show an increasing hydrogen concentration in the head sections of the screw, particularly along the outer surface where the gas flux enters. The final circular section corresponds to the shank: the concentration there remains zero, as the model assumes the shank is protected, while the head is exposed.

\begin{center}
\begin{minipage}{0.3\textwidth}
  \centering
  \textbf{0.08 cm}\\
  \textbf{0h} \includegraphics[height=53pt, trim=175 0 30 90, clip]{VITEEEEEE/0_008.jpg}\\
  \textbf{1h} \includegraphics[height=53pt, trim=175 0 30 90, clip]{VITEEEEEE/1_008.jpg}\\
  \textbf{2h} \includegraphics[height=53pt, trim=175 0 30 90, clip]{VITEEEEEE/2_008.jpg}\\
  \textbf{3h} \includegraphics[height=53pt, trim=175 0 30 90, clip]{VITEEEEEE/3_008.jpg}\\
  \textbf{4h} \includegraphics[height=53pt, trim=175 0 30 90, clip]{VITEEEEEE/4_008.jpg}\\
  \textbf{5h} \includegraphics[height=53pt, trim=175 0 30 90, clip]{VITEEEEEE/5_008.jpg}\\
  \textbf{6h} \includegraphics[height=53pt, trim=175 0 30 90, clip]{VITEEEEEE/6_008.jpg}\\
  \textbf{7h} \includegraphics[height=53pt, trim=175 0 30 90, clip]{VITEEEEEE/7_008.jpg}\\
  \textbf{8h} \includegraphics[height=53pt, trim=175 0 30 90, clip]{VITEEEEEE/8_008.jpg}\\
  \textbf{9h} \includegraphics[height=53pt, trim=175 0 30 90, clip]{VITEEEEEE/9_008.jpg}\\
  \textbf{10h} \includegraphics[height=53pt, trim=175 0 30 90, clip]{VITEEEEEE/10_008.jpg}
\end{minipage}\hfill
\begin{minipage}{0.2\textwidth}
  \centering
  \textbf{0.12 cm}\\
  \includegraphics[height=53pt, trim=175 0 30 90, clip]{VITEEEEEE/0_012.jpg}\\
  \includegraphics[height=53pt, trim=175 0 30 90, clip]{VITEEEEEE/1_012.jpg}\\
  \includegraphics[height=53pt, trim=175 0 30 90, clip]{VITEEEEEE/2_012.jpg}\\
  \includegraphics[height=53pt, trim=175 0 30 90, clip]{VITEEEEEE/3_012.jpg}\\
  \includegraphics[height=53pt, trim=175 0 30 90, clip]{VITEEEEEE/4_012.jpg}\\
  \includegraphics[height=53pt, trim=175 0 30 90, clip]{VITEEEEEE/5_012.jpg}\\
  \includegraphics[height=53pt, trim=175 0 30 90, clip]{VITEEEEEE/6_012.jpg}\\
  \includegraphics[height=53pt, trim=175 0 30 90, clip]{VITEEEEEE/7_012.jpg}\\
  \includegraphics[height=53pt, trim=175 0 30 90, clip]{VITEEEEEE/8_012.jpg}\\
  \includegraphics[height=53pt, trim=175 0 30 90, clip]{VITEEEEEE/9_012.jpg}\\
  \includegraphics[height=53pt, trim=175 0 30 90, clip]{VITEEEEEE/10_012.jpg}
\end{minipage}\hfill
\begin{minipage}{0.2\textwidth}
  \centering
  \textbf{0.25 cm}\\
  \includegraphics[height=53pt, trim=175 0 30 90, clip]{VITEEEEEE/bolt_hydrogen_0.vtk_0.25.png}\\
  \includegraphics[height=53pt, trim=175 0 30 90, clip]{VITEEEEEE/bolt_hydrogen_1.vtk_0.25.png}\\
  \includegraphics[height=53pt, trim=175 0 30 90, clip]{VITEEEEEE/bolt_hydrogen_2.vtk_0.25.png}\\
  \includegraphics[height=53pt, trim=175 0 30 90, clip]{VITEEEEEE/bolt_hydrogen_3.vtk_0.25.png}\\
  \includegraphics[height=53pt, trim=175 0 30 90, clip]{VITEEEEEE/bolt_hydrogen_4.vtk_0.25.png}\\
  \includegraphics[height=53pt, trim=175 0 30 90, clip]{VITEEEEEE/bolt_hydrogen_5.vtk_0.25.png}\\
  \includegraphics[height=53pt, trim=175 0 30 90, clip]{VITEEEEEE/bolt_hydrogen_6.vtk_0.25.png}\\
  \includegraphics[height=53pt, trim=175 0 30 90, clip]{VITEEEEEE/bolt_hydrogen_7.vtk_0.25.png}\\
  \includegraphics[height=53pt, trim=175 0 30 90, clip]{VITEEEEEE/bolt_hydrogen_8.vtk_0.25.png}\\
  \includegraphics[height=53pt, trim=175 0 30 90, clip]{VITEEEEEE/bolt_hydrogen_9.vtk_0.25.png}\\
  \includegraphics[height=53pt, trim=175 0 30 90, clip]{VITEEEEEE/bolt_hydrogen_10.vtk_0.25.png}
\end{minipage}\hfill
\begin{minipage}{0.2\textwidth}
  \centering
  \textbf{3 cm}\\
  \includegraphics[height=53pt, trim=175 0 30 90, clip]{VITEEEEEE/0_3.jpg}\\
  \includegraphics[height=53pt, trim=175 0 30 90, clip]{VITEEEEEE/1_3.jpg}\\
  \includegraphics[height=53pt, trim=175 0 30 90, clip]{VITEEEEEE/2_3.jpg}\\
  \includegraphics[height=53pt, trim=175 0 30 90, clip]{VITEEEEEE/3_3.jpg}\\
  \includegraphics[height=53pt, trim=175 0 30 90, clip]{VITEEEEEE/4_3.jpg}\\
  \includegraphics[height=53pt, trim=175 0 30 90, clip]{VITEEEEEE/5_3.jpg}\\
  \includegraphics[height=53pt, trim=175 0 30 90, clip]{VITEEEEEE/6_3.jpg}\\
  \includegraphics[height=53pt, trim=175 0 30 90, clip]{VITEEEEEE/7_3.jpg}\\
  \includegraphics[height=53pt, trim=175 0 30 90, clip]{VITEEEEEE/8_3.jpg}\\
  \includegraphics[height=53pt, trim=175 0 30 90, clip]{VITEEEEEE/9_3.jpg}\\
  \includegraphics[height=53pt, trim=175 0 30 90, clip]{VITEEEEEE/10_3.jpg}
\end{minipage}
\end{center}

